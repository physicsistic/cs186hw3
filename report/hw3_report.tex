\documentclass[11pt]{article}
\usepackage{amsfonts}
\usepackage[yyyymmdd,hhmmss]{datetime}
\usepackage{amsmath}
\usepackage{cancel}
\usepackage{mathtools}
\usepackage{graphicx}
\usepackage{multirow}
\usepackage{caption}
\usepackage{subcaption}

\setlength{\pdfpagewidth}{8.5in}
\setlength{\pdfpageheight}{11in}
\textwidth 6.5in
\textheight 9in
\headheight 0.0in
\topmargin -.5in
\oddsidemargin 0.0in
\evensidemargin 0.0in

\title{\bf CS 186 - Homework 3}
\date{\today}
\author{Matthew Warshauer \& Ye Zhao}
\begin{document}
\maketitle
\section{Designing a bidding agent}
See implmentation in code distribution.
\section{Analysis of the GSP agents}
\subsection{Average Utility}
We run the simulation for either 5 truthful agents or 5 balanced budget (mewzybb) agents with seed 5 and 200 iterations. The total utility for the two scenarios are shown below
\begin{itemize}
\item Truthful Agent: $u_{tot}=\$1659.37$
\item Balanced Agent (mewzybb): $u_{tot}=\$3342.55$
\end{itemize}
There is a improvement of \$1683.18 or approximately 67.3\%.
\\
\\
We see that the balanced bidding agents result in a better total utility of the bidding. This is because truthfulness is not an equilbrium in the generalized second price auction, while balanced bidding is an equilibrium. Switching from all truthful to all balanced bidding is a pareto improvement.
\subsection{Truthful vs Mewzybb}
When
\section{Auction Design and Reserve Prices}
\section{Budget constraints}
\end{document}
